\documentclass[11pt,a4paper]{article}

% --- Packages ---
\usepackage[utf8]{inputenc}
\usepackage[T1]{fontenc}
\usepackage{amsmath, amssymb, amsthm, bm}
\usepackage{geometry}
\usepackage{hyperref}

\geometry{margin=1in}

% --- Custom Commands ---
\newcommand{\Hspace}{\mathbb{H}}
\newcommand{\Rspace}{\mathbb{R}}
\newcommand{\Cspace}{\mathbb{C}}
\newcommand{\Dirac}{\mathcal{D}}
\newcommand{\quat}[1]{\mathbf{#1}}

\title{Advanced Geometric Signal Processing: \\ Quaternionic Time Series, Hyperkähler Manifolds, and Dirac Operators}
\author{Analytical Synthesis}
\date{\today}

\begin{document}

\maketitle

\section{Quaternionic Time Series and Logarithmic Mapping}
A quaternionic time series $\quat{q}(t) \in \Hspace$ often represents rotational trajectories. To avoid the non-linearity of the unit sphere $S^3$, we utilize the logarithmic map to project the series into the tangent space $\mathfrak{sp}(1) \cong \Rspace^3$.

The mapping for a unit quaternion $\quat{q} = [\cos(\theta/2), \hat{n}\sin(\theta/2)]$ is:
\begin{equation}
\bm{\omega}(t) = \ln(\quat{q}(t)) = \begin{cases} \frac{\theta}{2}\hat{n} & \theta \neq 0 \\ \vec{0} & \theta = 0 \end{cases}
\end{equation}

This allows the application of the Dirac operator on a flat Euclidean vector space before re-projecting via the exponential map $\exp(\bm{\omega})$.

\section{Hyperkähler Geometry and the Dirac Operator}
A Hyperkähler manifold $M$ is defined by a triplet of complex structures $(I, J, K)$ such that the metric $g$ is Kähler with respect to each. In the context of signal vector spaces, this provides three distinct symplectic forms $\omega_I, \omega_J, \omega_K$.

\subsection{The Hyperkähler Dirac Operator}
The Dirac operator $\Dirac$ on a Hyperkähler manifold decomposes into operators associated with the quaternionic structures. Given a spinor field $\Psi$ representing the time-varying state:
\begin{equation}
\Dirac = \sum_{a=1}^{4n} e^a \nabla_{e_a} = \Dirac_I I + \Dirac_J J + \Dirac_K K
\end{equation}
Where $\Dirac^2 = \Delta$ (the Laplacian). For a time series on a vector space, $\Dirac \Psi = 0$ defines the condition for \textit{hypermonogenicity}, ensuring the signal is globally consistent across all three complex projections.

\section{The Moment Map Constraint}
For a time series subject to symmetry groups (e.g., $SU(2)$), the Hyperkähler moment map $\mu: M \to \Rspace^3 \otimes \mathfrak{g}^*$ ensures that the evolution of the vector space preserves the underlying geometric integrity:
\begin{equation}
\mu(\quat{q}) = (\langle \quat{q}, I\quat{q} \rangle, \langle \quat{q}, J\quat{q} \rangle, \langle \quat{q}, K\quat{q} \rangle)
\end{equation}

\section{Consolidated Bibliography}
\begin{enumerate}
    \item \textbf{Dantam, N. T. (2020).} \textit{Practical Exponential Coordinates Using Implicit Dual Quaternions}. Springer.
    \item \textbf{Hitchin, N. J., et al. (1987).} Hyperkähler metrics and supersymmetry. \textit{Comm. Math. Phys.}, 108(4).
    \item \textbf{Parker, J., et al. (2023).} Logarithm-based methods for interpolating quaternion time series. \textit{Mathematics}, 11(5).
    \item \textbf{Brackx, F., et al. (2006).} \textit{Clifford Analysis}. CRC Press.
    \item \textbf{Cotăescu, I. I. (2006).} Dirac operators on Kählerian manifolds. \textit{Facta Univ.}
    \item \textbf{Freed, D. S. (1999).} \textit{Five Lectures on Supersymmetry}. AMS.
\end{enumerate}

\end{document}
