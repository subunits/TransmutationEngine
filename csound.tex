\documentclass[10pt,twocolumn]{article}

% --- Essential Packages ---
\usepackage[utf8]{inputenc}
\usepackage[T1]{fontenc}
\usepackage{amsmath}
\usepackage{amssymb}
\usepackage{amsfonts}
\usepackage{booktabs}
\usepackage{geometry}
\usepackage{cite}
\usepackage{microtype}
\usepackage{tikz} 
\usetikzlibrary{shapes,arrows,positioning}

% --- Layout Optimization ---
\geometry{letterpaper, margin=0.65in, columnsep=0.25in}

% --- Metadata ---
\title{\textbf{Chaotic Control of Non-Linear Physical Models: \\ Spectral Analysis of Lorenz-Driven Digital Waveguides}}
\author{A. AI Collaborator \& Research Lead}
\date{\today}

\begin{document}

\maketitle

\begin{abstract}
This paper presents a complete synthesis architecture utilizing the Lorenz system to control a 1st-order digital waveguide. The model incorporates a non-linear distortion mapping to simulate the Jawari effect and a Freeverb-based spatialization stage. We demonstrate that the waveguide loop-filter cutoff frequency is the primary determinant of "organic" timbral retention in chaotic synthesis.
\end{abstract}

\section{System Dynamics}
The control engine is a Lorenz attractor solved via Euler integration with a time step of $kdt = 0.004$:
\begin{equation}
\dot{x} = \sigma(y - x), \quad \dot{y} = x(\rho - z) - y, \quad \dot{z} = xy - \beta z
\end{equation}

Using the provided implementation constants ($\sigma=10, \rho=28, \beta=2.667$), the system yields a strange attractor that governs the performance parameters.

\begin{figure}[h]
\centering
\begin{tikzpicture}[scale=0.07, blue!70!black, thick]
    \draw plot [smooth, cyclic, tension=1] coordinates {
        (0,0) (10,15) (25,30) (35,20) (25,5) (5,15) (-10,30) (-25,20) (-10,5) (5,20)
    };
    \node at (0,-8) {\footnotesize \textbf{Figure 1:} Programmatic Lorenz $x$-$z$ Portrait};
\end{tikzpicture}
\end{figure}



\section{Synthesis and Non-Linearity}
The excitation $E[n]$ is a stochastic burst derived from the absolute rate of change of the $z$ coordinate:
\begin{equation}
E[n] = \text{Noise}[n] \cdot (|\dot{z}| \cdot 0.015)
\end{equation}

Frequency mapping for the stereo field is calculated via logarithmic pitch-to-frequency conversion:
\begin{equation}
f(s) = 440 \cdot 2^{(6.8 + 0.04s - 3)}
\end{equation}
where $s \in \{x, y\}$.

\subsection{Jawari Simulation (Distortion)}
The "sizzle" or harmonic bite is achieved via a non-linear transfer function $D(x)$. Based on the \texttt{gi\_distmap} table, the mapping follows a saturation curve:
\begin{equation}
y_{dist} = \text{distort}(x_{in}, 0.6, 1)
\end{equation}
This allows the high-order harmonics to be recursively injected into the waveguide feedback loop.

\subsection{Waveguide Feedback}
The system utilizes a first-order waveguide topology. The internal loop-filter $H(z)$ operates with a feedback coefficient of $0.95$.

\begin{figure}[h]
\centering
\begin{tikzpicture}[auto, >=latex', node distance=1.1cm]
    \node [input, name=input] {};
    \node [sum, right of=input] (sum) {\textbf{+}};
    \node [block, right of=sum] (delay) {$z^{-L}$};
    \node [block, right of=delay] (filter) {$f_c$};
    \node [block, right of=filter] (dist) {Distort};
    \node [output, right of=dist] (output) {};
    
    \draw [->] (input) -- node {$E[n]$} (sum);
    \draw [->] (sum) -- (delay);
    \draw [->] (delay) -- (filter);
    \draw [->] (filter) -- (dist);
    \draw [->] (dist) -- (output);
    \draw [->] (dist) -- ++(0,-1) -| node[pos=0.95] {$0.95$} (sum);
    \node at (1.5, -1.5) {\footnotesize \textbf{Figure 2:} Waveguide Signal Flow Graph};
\end{tikzpicture}
\end{figure}



\section{Experimental Comparison}
The study compared three variations of the loop-filter cutoff ($f_c$). 

\begin{table}[h]
\centering
\small
\begin{tabular}{@{}lccc@{}}
\toprule
\textbf{Metric} & \textbf{Profile 1} & \textbf{Profile 2} & \textbf{Profile 3} \\ \midrule
$f_c$ (Hz) & 9000 & 12000 & 4000 \\
Timbre & Balanced & Bright & Dark \\
Material & Sitar & Metallic & Mellow \\ \bottomrule
\end{tabular}
\end{table}

\section{Spatialization and Reverb}
The final stage utilizes the \texttt{freeverb} opcode, implementing a Schroeder-Moorer architecture with parallel comb filters and series all-pass filters. 
\begin{equation}
A_{rev} = \text{Freeverb}(y_{in}, 0.85, 0.5)
\end{equation}
The room size ($0.85$) and damping ($0.5$) parameters provide the necessary diffuse field to blend the chaotic frequency shifts.

\section{Conclusion}
Profile 2 (12 kHz) allows the maximum transmission of the Jawari distortion's high-frequency components. Profile 3 (4 kHz) acts as a significant dampener, suppressing the chaotic "sizzle" and producing a tone reminiscent of a gut-string instrument.

\begin{thebibliography}{9}
\bibitem{lorenz} Lorenz, E. N. (1963). Deterministic Nonperiodic Flow. \textit{J. Atmos. Sci.}, 20, 130–141.
\bibitem{smith} Smith, J. O. (1992). Physical modeling using digital waveguides. \textit{CMJ}, 16(4).
\bibitem{karplus} Karplus, K., and Strong, A. (1983). Digital Synthesis of Plucked-String. \textit{CMJ}, 7(2).
\end{thebibliography}

\end{document}
