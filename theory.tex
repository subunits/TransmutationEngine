\documentclass[12pt,a4paper]{article}

% --- Packages ---
\usepackage[utf8]{inputenc}
\usepackage[T1]{fontenc}
\usepackage{amsmath, amssymb, amsthm, bm}
\usepackage{geometry}
\usepackage{xcolor}
\usepackage{hyperref}
\usepackage{titlesec}
\usepackage{enumitem}

% --- Page Setup ---
\geometry{margin=1in}
\hypersetup{colorlinks=true, linkcolor=blue, citecolor=blue, urlcolor=teal}
\titleformat{\section}{\large\bfseries}{\thesection}{1em}{}

% --- Math Definitions ---
\newcommand{\Hspace}{\mathbb{H}}
\newcommand{\Rspace}{\mathbb{R}}
\newcommand{\quatLog}{\ln_{\mathbb{H}}}
\newcommand{\quatExp}{\exp_{\mathbb{H}}}

% --- Title ---
\title{\textbf{Geometric Dynamics of Quaternionic Time Series}\\ 
\large Logarithmic Linearization on Hyperkähler Manifolds}
\author{Formal Research Synthesis}
\date{\today}

\begin{document}

\maketitle

\begin{abstract}
This paper establishes a unified framework for the analysis of time series on quaternionic vector spaces. We argue that traditional Euclidean treatment of 4D signals fails to preserve the topological constraints of the 3-sphere ($S^3$). By utilizing the Hyperkähler structure of quaternionic space, we derive a log-Euclidean framework that enables linear signal processing on the tangent Lie algebra while maintaining the isometric properties of the underlying manifold. This approach bridges the gap between high-level differential geometry and practical signal estimation.
\end{abstract}

\section{The Problem of Dimensionality and Curvature}
In vector-valued time series, we typically assume an underlying Euclidean metric. However, for signals representing orientations, multi-channel phases, or 3D rotations, the data resides on a curved manifold. Direct operations on quaternionic vectors $\mathbf{q}(t)$ lead to "norm drift" from the unit 3-sphere ($S^3$), requiring frequent re-normalization which introduces non-linear noise and artifacts in the signal's second-order statistics \cite{parker2023}. We propose treating the time series not as a set of discrete vectors, but as a continuous trajectory on a \textit{Hyperkähler manifold}.

\section{The Hyperkähler Foundation}
A Hyperkähler manifold is a Riemannian manifold $(M, g)$ equipped with a triplet of anticommuting complex structures $\{I, J, K\}$ such that:
\begin{equation}
I^2 = J^2 = K^2 = IJK = -1
\end{equation}
These structures are covariant constant with respect to the Levi-Civita connection ($\nabla I = \nabla J = \nabla K = 0$), providing a rigid geometric framework where parallel transport preserves the quaternionic algebraic relationships \cite{hitchin1987}.

\subsection{The Metric and Symplectic Forms}
The geometry is uniquely defined by three Kähler forms $\omega_I, \omega_J, \omega_K$. For any tangent vectors $X, Y$, the metric $g$ satisfies:
\begin{equation}
g(X, Y) = \omega_I(IX, Y) = \omega_J(JX, Y) = \omega_K(KX, Y)
\end{equation}
This implies the "energy" or variance of a quaternionic time series is preserved across three different orthogonal complex planes simultaneously. This provides the mathematical basis for multi-channel synchronization and phase-coherence analysis in 4D signal processing \cite{mazzotti2012}.



\section{The Log-Euclidean Framework}
To perform standard arithmetic on time series (such as moving averages or Kalman filtering), we must "flatten" the manifold. We use the Quaternionic Logarithmic Map to project the series from the manifold $S^3$ into the tangent space $\mathfrak{so}(3)$ at the identity \cite{dantam2020}.

\subsection{Logarithmic Mapping}
For a unit quaternion $q = \cos(\theta/2) + \mathbf{u}\sin(\theta/2) \in S^3$, where $\mathbf{u}$ is a unit vector in $\mathbb{R}^3$, the mapping is defined as:
\begin{equation}
\mathbf{v}(t) = \quatLog(q(t)) = \frac{\theta(t)}{2}\mathbf{u}(t)
\end{equation}
The resulting vector $\mathbf{v}(t) \in \Rspace^3$ linearizes the rotational trajectory. In this transformed domain, the time series behaves as a standard Euclidean signal, allowing for the application of autoregressive (AR) models and spectral analysis \cite{szczesna2019}.

\subsection{Geodesic Evolution}
The shortest path between two states $q_1$ and $q_2$ on the Hyperkähler manifold (the geodesic) is not a straight line in $\mathbb{R}^4$, but a Spherical Linear Interpolation (Slerp). In the log-domain, this is expressed as:
\begin{equation}
q(t) = q_1 \quatExp \left( \frac{t-t_1}{t_2-t_1} \quatLog(q_1^{-1} q_2) \right)
\end{equation}
This ensures the evolution of the time series respects the intrinsic curvature of the state space, preventing the "shortcut" errors common in naïve vector interpolation.



\section{Information Geometry and Potentials}
In the Hyperkähler framework, the distance between two time-steps is governed by a \textit{Hyperkähler potential} $\rho$. The metric $g$ is the Hessian of this potential:
\begin{equation}
g_{\alpha\bar{\beta}} = \frac{\partial^2 \rho}{\partial q^\alpha \partial \bar{q}^\beta}
\end{equation}
For time series analysis, $\rho$ acts as a Lyapunov-like function, where the "flow" of the signal represents a trajectory of minimal action or maximum information entropy \cite{szczesna2019}.

\section{Conclusion}
By embedding quaternionic time series within the Hyperkähler framework, we leverage a rigid algebraic structure that naturally models 3D rotations and multi-channel correlations. The Log-Euclidean mapping serves as a critical bridge, allowing researchers to utilize powerful linear processing tools without sacrificing the geometric integrity of the quaternionic space.

% --- References ---
\begin{thebibliography}{99}

\bibitem{dantam2020}
N. T. Dantam, "Practical Exponential Coordinates Using Implicit Dual Quaternions," \textit{Springer Proceedings in Advanced Robotics}, pp. 639–655, 2020.

\bibitem{hitchin1987}
N. J. Hitchin, A. Karlhede, U. Lindström, and M. Roček, "Hyperkähler metrics and supersymmetry," \textit{Communications in Mathematical Physics}, vol. 108, no. 4, pp. 535–589, 1987.

\bibitem{mazzotti2012}
A. Mazzotti, A. Sajeva, G. M. Menanno, A. Grandi, and E. Stucchi, "Application of quaternion algorithms for multicomponent data analysis: A review," \textit{International Geophysical Conference}, 2012.

\bibitem{parker2023}
J. Parker, D. Ibarra, and D. Ober, "Logarithm-based methods for interpolating quaternion time series," \textit{Mathematics}, vol. 11, no. 5, p. 1131, 2023.

\bibitem{szczesna2019}
A. Szczęsna, "Quaternion Entropy for Analysis of Gait Data," \textit{Entropy}, vol. 21, no. 1, p. 79, 2019.

\end{thebibliography}

\end{document}
