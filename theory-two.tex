\documentclass[12pt,a4paper]{article}

% --- Essential Packages ---
\usepackage[utf8]{inputenc}
\usepackage[T1]{fontenc}
\usepackage{amsmath, amssymb, amsthm, bm}
\usepackage{geometry}
\usepackage{xcolor}
\usepackage{hyperref}
\usepackage{titlesec}

% --- Page Configuration ---
\geometry{margin=1in}
\hypersetup{colorlinks=true, linkcolor=blue, citecolor=blue, urlcolor=teal}
\titleformat{\section}{\large\bfseries}{\thesection}{1em}{}

% --- Mathematical Notations ---
\newcommand{\Hspace}{\mathbb{H}}
\newcommand{\Rspace}{\mathbb{R}}
\newcommand{\quatLog}{\ln_{\mathbb{H}}}
\newcommand{\quatExp}{\exp_{\mathbb{H}}}
\newcommand{\Dirac}{\mathcal{D}}

% --- Document Metadata ---
\title{\textbf{Geometric Dynamics of Quaternionic Time Series}\\ 
\large Logarithmic Linearization and Dirac Operators on Hyperkähler Manifolds}
\author{Formal Research Synthesis}
\date{\today}

\begin{document}

\maketitle

\begin{abstract}
This paper establishes a unified framework for the analysis of time series on quaternionic vector spaces. Traditional Euclidean treatments of 4D signals fail to preserve the topological constraints of the 3-sphere ($S^3$). By utilizing the Hyperkähler structure of quaternionic space, we derive a log-Euclidean framework for trajectory linearization. Furthermore, we introduce the Dirac operator as the governing differential mechanism for hyperholomorphic signal flow, ensuring phase-coherence and geometric integrity in high-dimensional state estimation.
\end{abstract}

\section{Trajectory Curvature and Vector Constraints}
In multi-dimensional time series, the assumption of a Euclidean metric frequently leads to "norm drift" in signals representing orientations or phases. When processing quaternionic data $\mathbf{q}(t)$, standard linear arithmetic violates the unit-norm constraint of $S^3$. We resolve this by treating the series as a trajectory on a \textit{Hyperkähler manifold}, where algebraic constraints are embedded in the metric itself.

\section{The Hyperkähler Metric Foundation}
A Hyperkähler manifold $(M, g)$ is characterized by a triplet of anticommuting complex structures $\{I, J, K\}$ satisfying the quaternionic identity:
\begin{equation}
I^2 = J^2 = K^2 = IJK = -1
\end{equation}
The metric $g$ is compatible with these structures, ensuring that parallel transport preserves the internal phase relationships of the signal. The energy of the quaternionic time series is thus preserved across three orthogonal complex planes simultaneously.



\section{Log-Euclidean Linearization}
To apply filtering or spectral analysis, the series must be mapped to a flat tangent space $\mathfrak{so}(3)$. For a unit quaternion $q = \cos(\theta/2) + \mathbf{u}\sin(\theta/2)$, the logarithmic mapping is defined as:
\begin{equation}
\mathbf{v}(t) = \quatLog(q(t)) = \frac{\theta(t)}{2}\mathbf{u}(t)
\end{equation}
This linearizes the trajectory into $\Rspace^3$. Geodesic evolution (Slerp) between states $q_1$ and $q_2$ follows a "straight line" within this linearized space:
\begin{equation}
q(t) = q_1 \quatExp \left( \frac{t-t_1}{t_2-t_1} \quatLog(q_1^{-1} q_2) \right)
\end{equation}



\section{Dirac Dynamics and Signal Flow}
The Dirac operator $\Dirac$ governs the differential evolution of the signal on the manifold. It is the first-order operator compatible with the Hyperkähler structure:
\begin{equation}
\Dirac = I \nabla_I + J \nabla_J + K \nabla_K
\end{equation}
A series is considered "hyperholomorphic" if it satisfies $\Dirac\psi = 0$. This operator captures the "spin" of the signal, ensuring that the components ($i, j, k$) do not decouple under high-frequency noise, thus maintaining phase-coherence.



\section{Conclusion}
The integration of log-Euclidean mappings with Dirac dynamics on Hyperkähler manifolds provides a framework that is both computationally efficient and geometrically rigorous. This approach prevents numerical instability and provides a natural language for the complex phase behaviors of 4D time series.

% --- Bibliography ---
\begin{thebibliography}{99}
\bibitem{dantam2020} N. T. Dantam, "Practical Exponential Coordinates Using Implicit Dual Quaternions," \textit{Springer}, 2020.
\bibitem{hitchin1987} N. J. Hitchin et al., "Hyperkähler metrics and supersymmetry," \textit{Comm. Math. Phys.}, 1987.
\bibitem{mazzotti2012} A. Mazzotti et al., "Application of quaternion algorithms for multicomponent data analysis," \textit{Int. Geophys. Conf.}, 2012.
\bibitem{parker2023} J. Parker et al., "Logarithm-based methods for interpolating quaternion time series," \textit{Mathematics}, 2023.
\bibitem{szczesna2019} A. Szczęsna, "Quaternion Entropy for Analysis of Gait Data," \textit{Entropy}, 2019.
\end{thebibliography}

\end{document}
